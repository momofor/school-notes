\usepackage{fontspec}
\usepackage{mathbbol}
\usepackage{mathtools}
\usepackage{array}
\usepackage{booktabs}
\usepackage{hyperref,cleveref}
\usepackage{titlesec}
\usepackage{babel}
\usepackage{tkz-tab}
\usepackage{amsmath,amsfonts,amsthm,amssymb}
\usepackage{mdframed}
\usepackage{pgfplots}
\usepackage{tcolorbox}
\usepackage{float}
\usepackage{multicol}
\hypersetup{
	colorlinks = true,
	linkcolor=blue
}

\usepackage[
    protrusion=true,
    activate={true,nocompatibility},
    final,
    tracking=true,
    factor=1100]
	{microtype}
\SetTracking{encoding={*}, shape=sc}{40}

\usepackage[a4paper,bottom=0.75in,right=0.8in,left=0.8in]{geometry}

\usepgfplotslibrary{external}
\tikzexternalize[prefix=cache/]

\pgfplotsset{compat=1.18,width=15cm}
\renewcommand{\thesection}{\Roman{section}}
\titleformat{\section}
{\normalfont\Large\bfseries}{\thesection-}{.5em}{}

\titleformat{\subsection}
{\normalfont\large\bfseries}{\thesubsection-}{.5em}{}



\newcounter{questionNumber}
\setcounter{questionNumber}{0}
\newcommand{\qs}[1]{
	Problem Number: \arabic{questionNumber}

	{#1}
	\addtocounter{questionNumber}{1}
}


%% One of the nicest things about LaTeX is you can create custom macros. If  there is a long-ish expression that you will write often, it is nice to give it a shorter command.
%% For our common number systems.
\newcommand{\RR}{\mathbb{R}} %% The blackboard-bold R that you have seen used for real numbers is typeset by $\mathbb{R}$. This macro means that $\RR$ will yield the same result, and is much shorter to type.
\newcommand{\NN}{\mathbb{N}}
\newcommand{\ZZ}{\mathbb{Z}} 
\newcommand{\QQ}{\mathbb{Q}}

%% Your macros can even accept arguments. 
\newcommand\set[1]{\left\lbrace #1 \right\rbrace} %% In mathmode, if you write \set{STUFF}, then this will output {STUFF}, i.e. STUFF inside of a set
\newcommand\abs[1]{\left| #1 \right|} %% This will do the same but with vertical bars. I.e., \abs{STUFF} gives |STUFF|
\newcommand\parens[1]{\left( #1 \right)} %% Similar. \parens{STUFF} gives (STUFF)
\newcommand\brac[1]{\left[ #1 \right]} %% Similar. \brac{STUFF} gives [STUFF]
\newcommand\sol[1]{\begin{mdframed}
\emph{Solution.} #1
\end{mdframed}}
\newcommand\solproof[1]{\begin{mdframed}
\begin{proof} #1
\end{proof}
\end{mdframed}}
\newcommand{\definition}[1]{
\begin{tcolorbox}[colback=red!5!white,colframe=red!50!black,title=Définition]
#1
\end{tcolorbox}}
\newcommand{\property}[1]{
\begin{tcolorbox}[colback=red!5!white,colframe=blue!50!black,title=propriétés]
#1
\end{tcolorbox}}

%% A few more important commands:

%% You should start every proof with \begin{proof} and end it with \end{proof}.  
%%
%% Code inside single dollar signs will give in-line mathmode. I.e., $f(x) = x^2$ 
%% Code \[ \] will give mathmode centered on its own line.
%%
%% Other common commands:
%%	\begin{align*} and \end{align*} -- Good for multiline equations
%%	\begin{align} and \end{align} -- Same as above, but it will number the equations for easy reference
%%	\emph{italicized text here} and \textbf{bold text here} are also useful.
%%
%% Some very specific mathmode commands and their meanings:
%%	x \in A -- x is an element of A
%%	x \notin A -- x is not an element of A
%%	A \subseteq B -- A is a subset of B
%%	A \subsetneq B -- A is a proper subset of B
%%	x \equiv y \pmod{n} -- x is congruent to y mod n. 
%%	x \geq y and x \leq y -- Greater than or equal to and less than or equal to 
%%
%% You'll probably find lots of relevant commands in the question prompts. Also Google is your friend!
